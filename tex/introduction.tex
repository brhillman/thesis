\chapter{Introduction}
General circulation models (also referred to as ``global climate models'' or ``GCMs'') are tools used both to simulate future climate change and to aid in understanding of the climate system. Relevant to the former of these uses, the Intergovernmental Panel on Climate Change (IPCC) affirms that there is ``considerable confidence'' that GCMs provide credible estimates of future climate change \citep{ar4_ch8}. This confidence stems from the physical basis of these models as well as their ability to simulate past and present climate. However, considerable uncertainties and shortcomings remain in these models and in their projections of future climate change, particularly in their representation of clouds.

Clouds are of primary importance to simulation of climate as they affect both the water balance and the top of atmosphere energy balance, yet their representation in climate models leaves much to be desired. Cloud feedback processes have been identified as a primary contributor to inter-model differences in equilibrium climate sensitivity and in projections made by climate models \citep{cess_et_al_1990,colman_2003,stephens_2005,webb_et_al_2006,bony_et_al_2006,williams_and_webb_2009}. This is not too surprising, due to the nature of the problem in representing processes important for cloud formation and evolution in climate models. Many of these processes operate at scales that cannot be directly resolved with the current resolution of climate models. The common approach to this problem is to represent model grid-box mean statistics associated with these properties with parameterizations based on empirical or semi-empirical relationships. Differences in the formulation of these parameterizations can have varying results for simulation of clouds in present day climate, and can be expected to lead to different cloud responses to climate change that then feed back on the climate system. Studies have shown that the sensitivity of climate models depends on the way clouds are parameterized \citep{mitchell_et_al_1987,senior_and_mitchell_1993,le_treut_et_al_1994,fowler_and_randall_1994,ma_et_al_1994,liang_and_wang_1997,yao_and_del_genio_2002,zhang_2004,stainforth_et_al_2005,yokohata_et_al_2005}. The evaluation of the representation of clouds in climate models is then an important task, and necessary to build confidence in projections of future climate change. 

A seemingly straightforward approach to evaluating the performance of climate models is comparing modeled quantities to those observed in present day climate. Although even perfect simulation of present day climate does not necessarily imply that a given model will respond like the real world in a changing climate, poor simulation of present day climate raises doubts as to the credibility of projections of future climate change. Evaluation of simulations of present day climate against observations thus offers a necessary first order test of climate model performance.

Satellite remote sensing instruments provide observations of cloud properties at the global or near global coverage necessary for comparison with climate models. However, comparisons of this sort are challenging and can often times be misleading due to the fact that modeled cloud properties are very different from what is observable from space. Instead of directly measuring the geophysical quantities of interest that climate models output, passive remote sensing instruments rely on various retrieval techniques to infer similar quantities from measured radiances. Physical limitations of the instruments and in the retrieval processes can lead to large differences in the end quantities retrieved from these instruments and can cause large uncertainties in comparisons, challenging the evaluation of climate models \citep{marchand_et_al_2010,pincus_et_al_2011}. 

In an effort to alleviate some of these issues and enable more robust comparisons of observed and modeled cloud properties, an approach taken recently has been to attempt to mimic the way in which satellite remote sensing instruments would retrieve geophysical quantities from model-produced cloud and radiative properties. This has been accomplished through the development of software referred to as ``instrument simulators''. These simulators do not necessarily attempt to do complete forward modeling of the retrieval process, but often rather make simple assumptions based on the known limitations or idiosyncrasies of a particular instrument to put the model produced quantities into a form more directly comparable to the instrument retrieved properties. This can be interpreted as a means to account for known uncertainties in the satellite retrievals to remove ambiguities in comparisons between models and satellite retrieval products.

Instrument simulation software was first developed for making comparisons with the International Satellite Cloud Climatology Project \citep[ISCCP;][]{rossow_and_schiffer_1999}, and the result is the ISCCP Clouds and Radiances Using SCOPS (ICARUS), or simply the ISCCP simulator \citep{klein_and_jakob_1999,webb_et_al_2001}. The ISCCP simulator has proven to be a useful model diagnostic tool, and has enabled a number of model evaluation studies \citep[e.g.,][]{norris_and_weaver_2001,lin_and_zhang_2004,zhang_et_al_2005,schmidt_et_al_2006,cole_et_al_2011}. The success of the ISCCP simulator in such comparisons has inspired the formulation of simulators for additional satellite remote sensing instruments. Simulation software has been developed for radar instruments such as CloudSat \citep{haynes_et_al_2007}, lidar instruments such as that on-board the Cloud-Aerosol Lidar and Infrared Pathfinder Satellite Observation satellite \citep[CALIPSO;][]{chepfer_et_al_2008}, the Multi-angle Imaging Spectroradiometer \citep[MISR;][]{marchand_and_ackerman_2010}, and the Moderate Resolution Imaging Spectroradiometer \citep[MODIS;][]{pincus_et_al_2011}. In order to facilitate the use of the wide range of instrument simulators now available for climate model evaluation, the Cloud Feedbacks Model Intercomparison Project \citep[CFMIP;][]{bony_et_al_2011} has pulled these simulators into a single software package with a common interface, the CFMIP Observation Simulator Package \citep[COSP;][]{bodas-salcedo_et_al_2011}.

In addition to facilitating comparisons between models and observations, instrument simulators can be useful in making intercomparisons between different models. Definitions of model cloud properties are not necessarily consistent across different models. A particularly troubling example of this is the inconsistency of the definition of cloud in the Community Atmosphere Model (CAM) versions 3 and 4. This inconsistency allows for the existence of cloud (particularly low-topped stratocumulus) in terms of the model diagnosed cloud fraction, but with no associated water content \citep{hannay_et_al_2009,medeiros_et_al_2011}. Such cloud has no impact on the radiation in the model, yet is still considered as cloud in the model output diagnostics. Inconsistencies such as these decrease the usefulness of inter-model comparisons using the model defined cloud properties. However, since the purpose of the instrument simulator is to simulate what the satellite instrument would see given the model defined cloud properties, the simulator provides a common framework for model intercomparisons by providing diagnostics of the radiatively important cloud properties as would be seen by the satellite instrument.

The use of instrument simulators has further use in assessing clouds in climate change scenarios. Although observations are not available to compare to modeled climate in such simulations, the diagnostics provided by the simulators show what changes (if any) can be expected in such a changed climate. The utility of circumventing inconsistent definitions of cloud properties and providing radiatively important diagnostics provides further justification for the use of simulators in the analysis of climate change simulations.

This study uses satellite remote sensing observations and the corresponding instrument simulator diagnostics from two global climate models to first perform a two-model intercomparison study to illustrate the utility of this technique. The same technique is then used to compare three successive versions of a particular model to assess improvements in new model versions. Instrument simulator diagnostics are then used to assess changes in clouds in perturbed climate simulations in two models in relation to biases identified in mean-state simulations with those models.
