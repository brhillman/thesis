%
% ----- title page
%
\Title{Evaluating clouds in global climate models using instrument simulators}
\Author{Benjamin R. Hillman}
\Year{2012}
\Program{\\UW Department of Atmospheric Sciences}
\Degree{Master of Science}
{\Degreetext{A thesis
  submitted in partial fulfillment of\\
  the requirements for the degree of}
 \titlepage
}

%
% ----- signature page (put real names in these)
%

\Chair{Thomas Ackerman}{Professor}{UW Atmospheric Sciences}

\Signature{Thomas Ackerman}
\Signature{Dargan Frierson}
\Signature{Cecilia Bitz}
\thesissignaturepage


% ----- quoteslip
%

% These are the real quote slips (choose one)
 
\thesisquoteslip

%  \doctoralquoteslip

%  \doctoralabstractquoteslip


%
% ----- abstract
%


\setcounter{page}{-1}
\abstract{%
Cloud feedback processes are recognized as being the largest source of inter-model differences in climate projections. This motivates a critical evaluation of the representation of clouds in global climate models. Satellite observations provide important information about cloud properties with global or near-global coverage, however comparisons between models and observations are often challenging due to limitations in the instrument retrievals and by differing spatial (and temporal) scales between models and observations. The Cloud Feedback Model Intercomparison Project (CFMIP) has developed an instrument simulator package to facilitate comparisons between models and satellite remote sensing observations. Many modeling centers are now incorporating this software into their climate models. Diagnostics from these simulators are used to evaluate the simulation of cloud properties in climate models in this study. Simulations of present-day climate using the Community Atmosphere Model (CAM) version 3 and the Geophysical Fluid Dynamics Laboratory (GFDL) Atmosphere Model (AM) version 2 are evaluated against observations using these simulator diagnostics. Relative performance of three successive versions of CAM (CAM3, CAM4, and CAM5) is evaluated using these diagnostics to compare each simulation with satellite remote sensing observations. Changes in cloud properties in response to a simplified climate change experiment are also explored using CAM3 and AM2.
}
 
%
% ----- contents & etc.
%
\tableofcontents
\listoffigures
\listoftables
 
%
% ----- glossary 
%
%\chapter*{Glossary}      % starred form omits the `chapter x'
%\addcontentsline{toc}{chapter}{Glossary}
%\thispagestyle{plain}
%%
%\begin{glossary}
%\item[COSP] CFMIP Observation Simulator Package
%\item[GCM] Global Circulation Model or Global Climate Model
%\item[ISCCP] International Satellite Cloud Climatology Project
%\item[MISR] Multi-angle Imaging Spectro-Radiometer
%\item[MODIS] Moderate Resolution Imaging Spectro-Radiometer
%\end{glossary}
 
%
% ----- acknowledgments
%
%\acknowledgments{% \vskip2pc
%  % {\narrower\noindent
%  The author wishes to express sincere appreciation to
%  University of Washington, where he has had the opportunity
%  to work with the \TeX\ formatting system,
%  and to the author of \TeX, Donald Knuth, {\it il miglior fabbro}.
%  % \par}
%}

%
% ----- dedication
%
%\dedication{\begin{center}to my dear wife, Joanna\end{center}}

%
% end of the preliminary pages
 
 
 
