\chapter{Summary and Future Work}
The representation of clouds in climate models is of critical importance to the simulation of climate. The use of satellite instrument simulators enables the type of robust and quantitative evaluation of the simulation of clouds in climate models against observations performed in this study.

Climate models are broadly constrained to achieve net energy balance at the top of the atmosphere. Although this is a necessary requirement to prevent simulated climates that are unstable and unphysical, it fails to constrain the individual effects that combine to produce the net balance, leaving room for potentially compensating errors in other quantities. In particular, although the net radiation balance is constrained, no constraints are placed on the total cloud amount, the vertical distribution of cloudiness, or on the cloud properties themselves. Quite often the cloud properties are used as tuning parameters and clouds are adjusted to enable the model to match the top of atmosphere energy balance. The presence of compensating biases in the cloud radiative properties in climate models has been suggested by others \citep[e.g.,][]{webb_et_al_2001,kay_et_al_2011} and has been demonstrated in this study for two widely used climate models. Recent studies have also pointed to similar biases in the representation of precipitation \citep{stephens_et_al_2010}.

The impact of compensating biases in mean-state simulation of clouds in climate models on cloud feedbacks have not yet been well documented. Using simple sea surface temperature perturbation experiments (which are typically used to diagnose feedbacks following \cite{cess_and_potter_1988}), the response of clouds to a prescribed surface warming has been assessed in this study using the same framework used to evaluate the simulation of clouds in present day climate against observations. This strategy facilitates connecting changes in climate change simulations with identified biases in simulations of the mean-state climate. The most substantial changes in the clouds in the perturbed climate seem closely related to biases identified in the mean-state climate. This has shown that the utility of the instruments simulators extends beyond evaluation of mean-state climate by providing a framework to assess changes associated with climate change consistent with evaluations of mean-state simulations. This also provides a common framework to evaluate a range of models and make inter-comparisons that is useful not only in evaluating the mean-state simulations of a range of models, but in comparing responses of different models to climate change using consistent diagnostics.

Although the presence of compensating biases has been identified in this study, quantification of the effect of individual biases on the top of atmosphere radiative fluxes has not been attempted. \cite{zelinka_thesis} has developed a set of radiative kernels that describe the sensitivity of the top of atmosphere radiative flux to changes in the different cloud types classified by the ISCCP joint histogram. By applying these kernels to biases identified in the ISCCP clouds types, the impact of these different biases on the top of atmosphere radiative fluxes can be quantified. This would allow an assessment of the radiative importance of different biases identified in the mean-state climate simulations. Similar kernels could be computed for cloud types identified by the MISR joint histogram as well.

Techniques used in this study lend themselves to model inter-comparisons of larger scale. Until now a robust evaluation of cloud properties across models has been difficult due to a lack of simple methods that relate the different information provided by satellite retrievals to model output. The ISCCP simulator has alleviated some of these issues, but the use of multiple observations to evaluate models increases the robustness of conclusions drawn from these sorts of comparisons. Comparisons using MISR observations also enable better evaluation of low-topped clouds, while MODIS observations provide a better estimate of mid and high-topped clouds. MODIS observations and the corresponding MODIS simulator also provide an opportunity to evaluate cloud microphysical properties such as effective particle size \citep{pincus_et_al_2011}, but observational uncertainty in some of these observed fields makes this difficult \citep{kay_et_al_2011}. The incorporation of COSP diagnostics in model simulations performed for the Climate Model Intercomparison Project Phase 5 \citep[CMIP5;][]{taylor_et_al_2011} promises to provide an opportunity to robustly evaluate the simulation of clouds in a wide range of climate models in a consistent manner using multiple observational datasets. This will further understanding of the strengths and weaknesses of climate models in simulating clouds, and hopefully drive future development efforts aimed at improving the representation of clouds in climate models.
